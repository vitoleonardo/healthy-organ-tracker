%%% Die folgende Zeile nicht ändern!
\section*{\ifthenelse{\equal{\sprache}{deutsch}}{Einleitung}{Abstract}}
%%% Zusammenfassung:


Diese Bachelorarbeit beschäftigt sich mit der semantischen Segmentierung von MRT Bildern im medizinischen Bereich. Im Jahr 2015 wurde bei 5 Millionen Menschen Krebs im Gastro-Intestinaltrakt festgestellt, wovon die Hälfte für eine Strahlentherapie in Frage kommen. Diese dauert in der Regel 1 bis 6 Wochen und findet täglich für 15 Minuten statt. Dabei soll den Krebszellen gezielt eine hohe Strahlendosis verabreicht werden, gleichzeitig ist es zu vermeiden nicht betroffene Organe dabei in Mitleidenschaft zu ziehen. Problematisch ist, dass  die Position der Organe und somit auch der Tumore von Tag zu Tag unterschiedlich ist. 

Dank Magnetresonanztomographen (MRT) und Linearbeschleunigern können Onkologen jedoch die genaue Position der Organe feststellen. Dies ist jedoch ein Zeitintensiver Prozess und dauert je nach Patienten zwischen 15 und 60 Minuten. Deep Learning könnte dabei helfen, diesen Prozess erheblich zu beschleunigen, um so Patienten zu entlasten und Ärzten die Chance zu geben mehreren Menschen im selben Zeitraum ihre Hilfe anzubieten. 

Das UW-Madison Carbone Cancer Center hat im Rahmen einer Challenge auf Kaggle.com anonymisierte Scans von über 150 Patienten zur Verfügung gestellt, in denen manuell der Dickdarm, Dünndarm und der Magen von Fachleuten segmentiert worden ist. Ziel ist es, ein Modell zu entwerfen welches gegeben den Trainingsdaten dazu in der Lage ist in Bildausschnitten von vorher ungesehenen Patienten die Lokalität der genannten Organe in Form von einer Maske zu bestimmen. 

Baseline dieser Arbeit bildet ein Convolutional Neural Network, der U-Net Segmentierungsarchitektur, welches auf Graustufenbildern antrainiert wurde. 

Auf einer detaillierten Datenanalyse aufbauend werden dann weitere Methoden angewendet, um die Performance zu verbessern. Es werden Trainigsdaten bereinigt und mit verschiedenen Eingabegrößen experimentiert. Ein weiterer Ansatz ist ein intelligenter Zuschnitt der Slices sowie die Verwendung von "2.5 dimensionalen" Daten, auf die ich später genauer eingehe. 

Zuletzt erfolgt eine Evaluation der Vorhersagungen sowie eine Gegenüberstellung der verschiedenen Modelle.