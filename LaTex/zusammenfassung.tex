
\section*{Danksagung}

An dieser Stelle möchte ich besonders meinen Eltern danken, mir das Studium möglich gemacht zu haben!

Ich danke Herrn Prof. Dr. Conrad und Herrn. Prof. Dr. Kollmann dafür diese Arbeit zu begutachten.

Außerdem danke ich Herrn Bogomasov, der mich während der gesamten Arbeit betreut hat und mir wertvolle Anregungen bot. 

\pagebreak
%%% Die folgende Zeile nicht ändern!
\section*{\ifthenelse{\equal{\sprache}{deutsch}}{Einleitung}{Abstract}}
%%% Zusammenfassung:


Diese Bachelorarbeit beschäftigt sich mit der semantischen Segmentierung von MRT Bildern im medizinischen Bereich. Im Jahr 2015 wurde bei 5 Millionen Menschen Krebs im Gastro-Intestinaltrakt festgestellt, davon kamen die Hälfte in Frage an einer Strahlentherapie im Ramen einer Behandlung teilzunehmen. Diese dauert in der Regel 1 bis 6 Wochen und findet täglich für 15 Minuten statt. Dabei soll dem Tumor gezielt eine hohe Strahlendosis verabreicht werden, gleichzeitig soll vermieden werden nicht betroffene Organe dabei in Mitleidenschaft zu ziehen. Problematisch ist, dass  die Position der Organe und somit auch der Tumore von Tag zu Tag unterschiedlich ist. 

Dank Magnetresonanztomographen (MRT) und Linearbeschleunigern können Onkologen die genaue Position der Organe feststellen. Dies ist jedoch ein zeitintensiver Prozess und dauert je nach Patienten zwischen 15 und 60 Minuten. An dieser Stelle könnten Methoden aus dem Bereich des maschinellen Lernens Anwendung finden.

State-of-the-Art Deep Learning Modelle sind bereits dazu in der Lage, MRT-Abbilder anderer Körperregionen pixelgenau zu segmentieren - ein Transfer dieser Modelle auf die Organe des Magen-Darm-Trakts könnte die klassische Behandlung erheblich beschleunigen. So würden Patienten sowie Ärzte entlastet werden.

Das UW-Madison Carbone Cancer Center hat im Rahmen eines Wettbewerbs auf Kaggle.com anonymisierte MRT-Aufnahmen von über 150 Patienten zur Verfügung gestellt, in denen manuell der Dickdarm, Dünndarm und der Magen von Fachleuten segmentiert worden ist. Ziel ist es, ein Modell zu entwerfen welches gegeben den Trainingsdaten dazu in der Lage ist in Bildausschnitten von vorher ungesehenen Patienten die Lokalität der genannten Organe vorherzusagen. 

Baseline dieser Arbeit bildet ein Convolutional Neural Network (CNN), die U-Net Segmentierungsarchitektur. Auf einer detaillierten Analyse der Daten aufbauend wird mit verschiedenen Erweiterungen experimentiert, mit dem Ziel eine genauere Vorhersage zu erreichen. Falsch annotierte Daten werden entfernt, Pixelverhältnisse von Vorder- und Hintergrund normalisiert und die Dimensionen der Eingabebilder werden erweitert.